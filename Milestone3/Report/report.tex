\documentclass{scrreport}
\usepackage[utf8]{inputenc}
\usepackage{float} 
\usepackage{graphicx}
\usepackage{hyperref}

% ---- table / layout helpers ------------------
\usepackage{booktabs}   % \toprule, \midrule, \bottomrule
\usepackage{tabularx}   % column type X for width-flexible tables
\usepackage{array}      % extended column formatting
\usepackage{ragged2e}   % \raggedright inside tabularx
\usepackage{float}      % H float placement specifier
\usepackage{caption}    % nicer table captions / spacing
\usepackage{amsmath} 

% Add any additional packages you may need here

\renewcommand*{\thesection}{\Roman{section}}
\renewcommand*{\thesubsection}{\arabic{subsection}}
\renewcommand*{\thesubsubsection}{\roman{subsubsection}}

\title{Security Oracles - a new active approach to Smart Contracts Security}
\subtitle{Milestone Report}

\author{Team Sentinel}
\date{\today}

\graphicspath{Images/}


\begin{document}


\maketitle

\tableofcontents

\newpage
\section{Introduction}
Security challenges in decentralized systems continue to evolve, necessitating robust and adaptive threat detection strategies. This project set out to demonstrate how Security Oracles can enable Cardano smart contracts to adapt their execution logic in response to real-time threats. These oracles assume the existence of trusted and reliable detection systems capable of identifying such threats off-chain.

While the detailed design and implementation of detection engines fall outside the scope of this project, the purpose of this report is to explore potential strategies and methodologies for building them. We begin by reviewing the landscape of threat detection approaches in Web3, identifying those most compatible with Cardano’s extended UTXO (eUTXO) model. We then propose several viable paths the Cardano ecosystem can pursue to incentivize the development and sustained operation of such infrastructure.

Through this exploration, we aim to lay the groundwork for ensuring that Cardano remains resilient in the face of evolving attack vectors—by equipping its contracts with the tools to respond intelligently and autonomously to credible off-chain threat intelligence.


\section{The Threat-Detection Landscape}
Off-chain security monitoring has matured rapidly, producing a spectrum of service models that differ
in trust assumptions, scalability, and economic incentives.  
From a Cardano perspective, five architectural \emph{archetypes} dominate today’s Web 3.0 ecosystem.

\begin{table}[H]
\centering
\caption{Dominant archetypes for off-chain security monitoring}
\begin{tabularx}{\textwidth}{>{\raggedright\arraybackslash}p{0.6cm}%
                          >{\raggedright\arraybackslash}p{2.6cm}%
                          >{\raggedright\arraybackslash}X
                          >{\raggedright\arraybackslash}p{3cm}
                          >{\raggedright\arraybackslash}p{3cm}}
\toprule
\# & Archetype & How it works & Representative solutions & Typical output \\ \midrule
1 & \textbf{Centralized SaaS analytics} &
    Single vendor ingests full-node data (often multi-chain), enriches with heuristics/ML, and exposes
    APIs. &
    Chainalysis \textit{Reactor}, CertiK \textit{SkyInsights}, TRM Labs, Hypernative &
    Risk scores, tagged addresses, push alerts \\ \addlinespace[0.3em]

2 & \textbf{Decentralized monitoring networks} &
    Node operators stake a work token and run community-authored ``detection bots''; social–economic
    consensus (reputation + slashing) enforces quality. &
    Forta, Pessimism (OP Stack), Caldera \textit{Watcher} &
    Signed JSON alerts, on-chain attestations \\ \addlinespace[0.3em]

3 & \textbf{Oracle-integrated feeds} &
    Threat intelligence delivered through existing oracle rails; validator scripts verify signatures
    in-contract. &
    Chainlink \textit{Data Streams}, Supra, RedStone \textit{Security Feeds} &
    On-chain data points, e.g.\ \texttt{threat\_level = 3} \\ \addlinespace[0.3em]

4 & \textbf{Mempool / pre-chain firewalls} &
    RPC middleware inspects pending transactions, scores risk, blocks or timelocks suspicious submissions
    before inclusion. &
    Forta \textit{Firewall}, Blockfence \textit{Guardian}, Hexagate &
    Allow / deny verdicts \\ \addlinespace[0.3em]

5 & \textbf{Protocol-native runtime monitors} &
    A protocol team embeds an off-chain daemon that tracks invariants and can pause contracts via admin
    keys or governance. &
    MakerDAO \textit{Circuit Breaker}, Aave \textit{Guardian}, Lido \textit{Safety Module} &
    Admin transactions (pause, rate-limit) \\ \bottomrule
\end{tabularx}
\end{table}

\subsection{Core functional building blocks}
\begin{enumerate}
  \item \textbf{Data ingestion layer} — archive/full nodes, mempool listeners, external APIs.
  \item \textbf{Analytics \& detection} — static rules, heuristics, machine-learning anomaly models,
        graph queries, transaction simulation.
  \item \textbf{Alert signing \& transport} — ECDSA/EdDSA signatures, threshold schemes,
        oracle aggregation.
  \item \textbf{On-chain response hooks} — pausable contracts, UTXO-consuming ``watch-dog'' scripts,
        roll-up settlement guards.
\end{enumerate}

\subsection{Key comparison dimensions}

\begin{table}[H]
\centering
\caption{Important evaluation dimensions and current industry ranges}
\begin{tabularx}{\textwidth}{>{\raggedright\arraybackslash}p{3.3cm}%
                          >{\raggedright\arraybackslash}X
                          >{\raggedright\arraybackslash}p{5.2cm}}
\toprule
Dimension & Why it matters & Current industry range \\ \midrule
\textbf{Chain coverage} & Cross-chain exploits propagate quickly; multi-chain visibility is mandatory. &
  Single-chain (e.g.\ Solana-only) $\rightarrow$ 20\,+ chains \\ \addlinespace[0.3em]

\textbf{Data freshness} & Shorter detection lag reduces economic damage. &
  $\sim$250 ms (mempool firewalls) $\rightarrow$ $\approx$1 block delay \\ \addlinespace[0.3em]

\textbf{Sustainable throughput} & Determines the number of transactions/events an engine can analyse
without bottlenecks. &
  800 tx/s (lightweight mempool filters) $\rightarrow$ $<10$ tx/s (heavy ML inference) \\ \addlinespace[0.3em]

\textbf{Alert authenticity} & Contracts must trust that alerts are untampered. &
  Vendor API keys $\rightarrow$ multi-sig oracle attestations \\ \addlinespace[0.3em]

\textbf{Operator incentives} & Drives service longevity and quality. &
  Subscription fees, token staking rewards, DAO grants \\ \bottomrule
\end{tabularx}
\end{table}

\subsection{Trends and gaps relevant to Cardano}
\begin{itemize}
  \item \emph{From monolithic SaaS to modular networks} — teams pair deep forensic SaaS with
        decentralized real-time alert feeds.
  \item \emph{Mempool-aware blocking becomes standard} — UTXO ecosystems require analogous
        ``transaction guardians.''  
  \item \emph{Convergence on signed-alert formats} — e.g.\ W3C Verifiable Credentials, Sign Protocol.  
  \item \emph{Economic experimentation} — work-token and pay-per-trigger models replace flat
        subscriptions.  
  \item \emph{Under-explored UTXO support} — most engines assume account-based state; adapting them to
        eUTXO presents an open R\&D opportunity.
\end{itemize}

This landscape sets the stage for the next section, in which we distil \emph{three concrete approaches}
tailored to Cardano, complete with throughput, freshness, and response-time benchmarks required to
deliver practical Security Oracles.


\section{Cardano-Focused Approaches to Off-Chain Security Monitoring}
In light of the landscape analysis, we identify \emph{three} concrete aproaches to off-chain
security monitoring that map cleanly onto Cardano’s extended-UTXO (eUTXO) model 

\begin{table}[H]
\centering
\caption{Benchmarks for Cardano-relevant monitoring approaches}
\begin{tabularx}{\textwidth}{>{\raggedright\arraybackslash}p{3.2cm}%
                          >{\raggedright\arraybackslash}X%
                          >{\raggedright\arraybackslash}p{2cm}%
                          >{\raggedright\arraybackslash}p{2.2cm}%
                          >{\raggedright\arraybackslash}p{2.3cm}}
\toprule
Approach & Core idea & Sust.\ throughput\footnotemark & Data freshness & Target response time \\ \midrule
\textbf{A1. Community \mbox{Staked} Detection Network} &
  Forta-style network; operators stake ADA or a work token and run open-source “Cardano
  bots” that stream block and mempool data via Ogmios. &
  $\sim$50\,tx/s per node;
  horizontal scaling with $n$ nodes &
  $< 1$ block ($\approx 20$ s) &
  Submit mitigation tx in
  $< 40$ s total \\ \addlinespace[0.3em]

\textbf{A2. Oracle Attestation Layer} &
  Chainlink-like oracle feed that publishes signed threat scores
  every block; contracts
  verify the signature in-script and adapt logic. &
  Negligible on-chain load;
  oracle must handle $\sim$3\,MB/day data uplink &
  1–2 blocks
  (20–40 s) &
  Contract branch executed in the same block that consumes
  the attestation \\ \addlinespace[0.3em]

\textbf{A3. Treasury-Funded \mbox{Watch-Dog} Service} &
  A central analytics provider (or consortium) funded by DAO/treasury
  grants runs deep ML models off-chain and submits watchdog transactions
  that pause/escrow funds when anomalies are detected. &
  Up to full-chain bandwidth
  ($\approx$250\,tx/s) on provider side &
  $<\!5$ s event ingest;
  1 block chain latency &
  Pause UTXO or flip datum
  in $< 30$ s \\ \bottomrule
\end{tabularx}
\end{table}

\footnotetext{Throughput figures assume the current main-net parameter
(\texttt{maxBlockBodySize} $\approx$\,88 kB) and typical
block intervals of 20 s.}

\subsection{A1 – Community Staked Detection Network}

\paragraph{Architecture.}
\begin{itemize}
  \item Each operator stakes ADA (or a work token) into a \texttt{MonitorRegistry} contract.
  \item Nodes consume Ogmios/WebSocket streams, run detection bots, and sign
        alerts with Ed25519 keys registered on-chain.
  \item A lightweight on-chain aggregator contract accepts the \emph{first} $k$ matching
        signatures (threshold scheme) and mints a “\texttt{ThreatAlert}” UTXO that
        downstream contracts can inspect.
\end{itemize}

\paragraph{Advantages.}
Decentralised, censorship-resistant, aligns with ADA staking culture,
can scale horizontally, and slashing discourages false alerts.

\paragraph{Challenges.}
\emph{Bootstrap problem} (critical-mass of honest operators), higher
complexity for threshold aggregation, and 1-block latency may be too slow
for flash-loan-style attacks.

\subsection{A2 – Oracle Attestation Layer}

\paragraph{Architecture.}
A consortium of oracle nodes (e.g.\ re-used Chainlink infrastructure) 
pushes a signed JSON blob~$\langle\text{tx\_hash},\,
\text{threat\_score}\rangle_{\text{sig}}$ into a reference script every
block.  Contracts validate the signature inside their spending scripts and
reject or re-route value when \texttt{threat\_score} exceeds a threshold.

\paragraph{Advantages.}
Minimal on-chain load, deterministic gas costs, and direct
contract-level branching compatible with eUTXO statelessness.

\paragraph{Challenges.}
Relies on oracle honesty (Byzantine assumptions),
still incurs 1–2-block lag, and requires every dApp to integrate the
verification logic.

\subsection{A3 – Treasury-Funded Watch-Dog Service}

\paragraph{Architecture.}
A protocol treasury funds an ML-powered SOC (Security-Operations-Centre)
that:
\begin{enumerate}
  \item Streams full-chain data, external intel, and mempool submissions.
  \item Runs heavy simulations (e.g.\ MBO “dry-run”) to detect exploits.
  \item Signs and submits a privileged “pause” or “freeze” transaction
        (multi-sig with protocol guardians) that flips a datum or spends a
        control UTXO.
\end{enumerate}

\paragraph{Advantages.}
Deep analytics capabilities, sub-block detection latency ($<\!5$ s),
clear accountability to the funding DAO.

\paragraph{Challenges.}
Centralisation risk, long-term OPEX burden, and possible over-reach
(\emph{false positives} causing unnecessary contract halts).

\subsection*{Summary}
These three approaches span the decentralisation spectrum—from fully
community-run to protocol-controlled—while meeting benchmark targets
for throughput, freshness, and reaction time on Cardano.  Subsequent work
can combine elements (e.g.\ oracle attestations \emph{plus} a staked network)
to strike bespoke trade-offs for individual dApps.


\section{Economic and Governance Models for Cardano Security-Monitoring}

Reliable monitoring infrastructure will not emerge without clear,
self-reinforcing incentives.  Below we evaluate four business models
that could sustain off-chain threat-detection services in the Cardano
ecosystem.  Each model is assessed for stakeholder alignment,
long-term sustainability, and practicality under current Cardano
governance and treasury mechanisms.

\begin{table}[H]
\centering
\caption{Economic models for funding and operating Cardano threat-detection systems}
\begin{tabularx}{\textwidth}{>{\raggedright\arraybackslash}p{2.8cm}%
                          >{\raggedright\arraybackslash}X%
                          >{\raggedright\arraybackslash}p{2.3cm}%
                          >{\raggedright\arraybackslash}p{2.8cm}%
                          >{\raggedright\arraybackslash}p{2cm}}
\toprule
Model & Core funding logic & Primary stakeholders & Incentive mechanism & Sustainability outlook \\ \midrule
\textbf{M1.~Treasury Grants \& DAO Bounties} &
  Catalyst-style grants or protocol-specific DAOs allocate budgets for
  “Security Operations” milestones; recurring bounties for high-severity alerts. &
  Cardano Treasury, dApp-specific DAOs, security teams, community voters &
  Reputation + milestone payouts; bounty bonuses for critical saves &
  Medium: depends on continuous governance engagement and treasury health \\ \addlinespace[0.3em]

\textbf{M2.~Work-Token \mbox{Staking} Network} &
  Operators stake a \texttt{MONITOR} token; rewards stream from protocol-set %
  “monitoring fee” charged per transaction.  Slashing for invalid alerts. &
  Node operators, token holders, dApp developers, delegators &
  Block-by-block emission + fee share proportional to stake and SLA score &
  High if fee rate auto-adjusts to network load; boot-strap via airdrop \\ \addlinespace[0.3em]

\textbf{M3.~Pay-per-Trigger Service Fee} &
  dApps earmark a tiny fraction (e.g.\ 2–5 bps) of each transaction or %
  vault harvest into an escrow.  Smart contract pays \emph{only} when a %
  signed alert is consumed. &
  dApp treasuries, integrators, insurance underwriters, oracle operators &
  Direct revenue linked to successful detection; no work = no pay &
  Moderate–High: cost scales with delivered value; requires robust %
  false-alert arbitration \\ \addlinespace[0.3em]

\textbf{M4.~Enterprise SaaS + Insurance Wrap} &
  Central SOC (Security-Operations-Centre) sells subscription tiers to %
  exchanges, wallets, and institutional vaults; %
  optionally bundles on-chain exploit insurance. &
  SOC vendor, custodians, institutional funds, reinsurers &
  Flat monthly fee + %
  performance rebate; insurance premium offsets black-swan risk &
  High for enterprise segment; weaker community externalities \\ \bottomrule
\end{tabularx}
\end{table}

\subsection{M1 – Treasury Grants and DAO Bounties}

\emph{Mechanics.}  Funding flows from Catalyst funds or
protocol-level DAOs to open-source monitoring teams via milestone-based
contracts.  Additional performance bounties can be crowdsourced: each dApp
escrows a small ADA pool; the first detector to publish a valid alert
that averts a loss receives the bounty.

\emph{Strengths.}  Leverages Cardano’s established governance rails;
no need for new tokens.  Aligns community good-will with concrete
deliverables.

\emph{Weaknesses.}  Lumpy cash-flow; subject to governance fatigue.
Response time to new threats depends on DAO voting cadence.

\subsection{M2 – Work-Token Staking Network}

\emph{Mechanics.}  Mirrors Forta: operators bond \texttt{MONITOR} tokens;
a monitoring fee (e.g.\ 0.02 ADA/tx) is hard-coded in a CIP and routed to
a \texttt{RewardSplitter} script that pays stakers by SLA score.  Malicious or
low-quality nodes are slashed.

\emph{Strengths.}  Continuous, programmatic rewards; skin-in-the-game via
slashing; permissionless entry promotes decentralisation.

\emph{Weaknesses.}  Requires new token economics and potentially divisive
governance to set the base fee; initial cold-start of token liquidity.

\subsection{M3 – Pay-per-Trigger Service Fee}

\emph{Mechanics.}  Each integrator contract escrows funds into a
\texttt{TriggerEscrow} datum.  When a signed alert is consumed by the
contract, the validator releases the pre-negotiated fee to the alert
publisher.  Arbitration layer adjudicates disputed or false alerts.

\emph{Strengths.}  Payment strictly proportional to value delivered;
no idle overhead; easy to pilot on a per-dApp basis.

\emph{Weaknesses.}  Detect-then-pay loop demands robust on-chain
arbitration for false positives; revenue can be bursty.

\subsection{M4 – Enterprise SaaS with Insurance}

\emph{Mechanics.}  A regulated SOC offers API keys, dashboard access, and
signed on-chain feeds to large custodians, wallets, and funds.
An optional insurance rider pays out if clients suffer a loss not
prevented by the SOC’s alerts.

\emph{Strengths.}  Clear legal contracts, guaranteed SLAs, predictable
revenue, ability to fund heavy ML pipelines.

\emph{Weaknesses.}  Centralisation; limited coverage of long-tail dApps;
trust anchored in legal recourse rather than on-chain slashing.

\subsection*{Comparative Insights}

\begin{itemize}
  \item \textbf{Stake-based networks (M2)} offer the best blend of continuous
        incentives and decentralisation, but need careful fee-rate governance.
  \item \textbf{Pay-per-trigger (M3)} aligns payment with actual defence
        value—ideal for high-TVL vaults that can tolerate micro-fees.
  \item \textbf{Treasury grants (M1)} excel for boot-strapping open tooling,
        whereas \textbf{enterprise SaaS (M4)} fills compliance-oriented niches.
\end{itemize}

A layered strategy—Catalyst grants for initial R\&D, followed by a
work-token network supplemented with per-dApp trigger fees—can
share costs fairly across the ecosystem while preserving Cardano’s ethos
of decentralised, formally verified security infrastructure.



\end{document}